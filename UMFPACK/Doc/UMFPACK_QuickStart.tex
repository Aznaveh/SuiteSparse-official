%-------------------------------------------------------------------------------
% The QuickStart.tex file.
%-------------------------------------------------------------------------------

\documentclass[11pt]{article}
\batchmode

\newcommand{\m}[1]{{\bf{#1}}}       % for matrices and vectors
\newcommand{\tr}{^{\sf T}}          % transpose

\topmargin 0in
\textheight 8.5in
\oddsidemargin 0pt
\evensidemargin 0pt
\textwidth 6.5in

\begin{document}

\author{Timothy A. Davis \\
DrTimothyAldenDavis@gmail.com, http://www.suitesparse.com}
\title{UMFPACK Quick Start Guide}
% version of SuiteSparse/UMFPACK
\date{VERSION 6.0.1, Nov 12, 2022}

\maketitle

%-------------------------------------------------------------------------------
\begin{abstract}
    UMFPACK is a set of routines for solving unsymmetric sparse linear
    systems, $\m{Ax}=\m{b}$, using the Unsymmetric-pattern MultiFrontal method
    and direct sparse LU factorization.  It is written in ANSI/ISO C, with a
    MATLAB interface.  UMFPACK relies on the Level-3
    Basic Linear Algebra Subprograms (dense matrix multiply) for its
    performance.
    This is a ``quick start'' guide for Unix/Linux users of the C interface.
\end{abstract}
%-------------------------------------------------------------------------------

UMFPACK, Copyright\copyright 2005-2023, Timothy A. Davis, All Rights Reserved.

SPDX-License-Identifier: GPL-2.0+

See http://www.suitesparse.com
for the code and full documentation.

%-------------------------------------------------------------------------------
\section{Overview}
%-------------------------------------------------------------------------------

UMFPACK is a set of routines for solving systems of linear
equations, $\m{Ax}=\m{b}$, when $\m{A}$ is sparse and unsymmetric.
The sparse matrix $\m{A}$ can be square or rectangular, singular
or non-singular, and real or complex (or any combination).  Only square
matrices $\m{A}$ can be used to solve $\m{Ax}=\m{b}$ or related systems.
Rectangular matrices can only be factorized.

UMFPACK is a built-in routine in MATLAB used by the forward and
backslash operator, and the {\tt lu} routine.
The following is a short
introduction to Unix users of the C interface of UMFPACK.

%-------------------------------------------------------------------------------

The C-callable UMFPACK library consists of 32 user-callable routines and one
include file.  Twenty-eight of the routines come in four versions, with
different sizes of integers and for real or complex floating-point numbers.
This Quick Start Guide assumes you are working with real matrices
(not complex) and with {\tt int}'s as integers (not {\tt long}'s).
Refer to the User Guide for information about the complex and
long integer versions.  The include file {\tt umfpack.h}
must be included in any C program that uses UMFPACK.

For more details, see:
{\em A column pre-ordering strategy for the unsymmetric-pattern multifrontal method},
Davis, T. A.,
ACM Trans. Math. Software, vol 30. no 2, 2004, pp. 165-195, and
{\em Algorithm 832:  {UMFPACK}, an unsymmetric-pattern multifrontal method},
same issue, pp. 196-199.

%-------------------------------------------------------------------------------
\section{Primary routines, and a simple example}
%-------------------------------------------------------------------------------

Five primary UMFPACK routines are required to factorize $\m{A}$ or
solve $\m{Ax}=\m{b}$.  An overview of the primary features of the routines
is given in Section~\ref{Primary}.
Additional routines are available for passing a different column ordering
to UMFPACK, changing default parameters, manipulating sparse matrices,
getting the LU factors, save and loading the LU factors from a file,
computing the determinant,
and reporting results.  See the User Guide for more information.

\begin{itemize}
\item {\tt umfpack\_di\_symbolic}:

    Pre-orders the columns of $\m{A}$ to reduce fill-in and performs a
    symbolic analysis.
    Returns an opaque {\tt Symbolic} object as a {\tt void *}
    pointer.  The object contains the symbolic analysis and is needed for the
    numerical factorization.

\item {\tt umfpack\_di\_numeric}:

    Numerically scales and then factorizes a sparse matrix
    $\m{PAQ}$, $\m{PRAQ}$, or $\m{PR}^{-1}\m{AQ}$ into the product $\m{LU}$,
    where
    $\m{P}$ and $\m{Q}$ are permutation matrices, $\m{R}$ is a diagonal
    matrix of scale factors, $\m{L}$ is lower triangular with unit diagonal,
    and $\m{U}$ is upper triangular.  Requires the
    symbolic ordering and analysis computed by {\tt umfpack\_di\_symbolic}.
    Returns an opaque {\tt Numeric} object as a
    {\tt void *} pointer.  The object contains the numerical factorization and
    is used by {\tt umfpack\_di\_solve}.

\item {\tt umfpack\_di\_solve}:

    Solves a sparse linear system ($\m{Ax}=\m{b}$, $\m{A}\tr\m{x}=\m{b}$, or
    systems involving just $\m{L}$ or $\m{U}$), using the numeric factorization
    computed by {\tt umfpack\_di\_numeric}.

\item {\tt umfpack\_di\_free\_symbolic}:

    Frees the {\tt Symbolic} object created by {\tt umfpack\_di\_symbolic}.

\item {\tt umfpack\_di\_free\_numeric}:

    Frees the {\tt Numeric} object created by {\tt umfpack\_di\_numeric}.

\end{itemize}

The matrix $\m{A}$ is represented in compressed column form, which is
identical to the sparse matrix representation used by MATLAB.  It consists
of three arrays, where the matrix is {\tt m}-by-{\tt n},
with {\tt nz} entries:

{\footnotesize
\begin{verbatim}
     int32_t Ap [n+1] ;
     int32_t Ai [nz] ;
     double Ax [nz] ;
\end{verbatim}
}

All nonzeros are entries, but an entry may be numerically zero.  The row indices
of entries in column {\tt j} are stored in
    {\tt Ai[Ap[j]} ... {\tt Ap[j+1]-1]}.
The corresponding numerical values are stored in
    {\tt Ax[Ap[j]} ... {\tt Ap[j+1]-1]}.

No duplicate row indices may be present, and the row indices in any given
column must be sorted in ascending order.  The first entry {\tt Ap[0]} must be
zero.  The total number of entries in the matrix is thus {\tt nz = Ap[n]}.
Except for the fact that extra zero entries can be included, there is thus a
unique compressed column representation of any given matrix $\m{A}$.

Here is a simple main program, {\tt umfpack\_simple.c}, that illustrates the
basic usage of UMFPACK.

{\footnotesize
\begin{verbatim}
    #include <stdio.h>
    #include "umfpack.h"

    int32_t n = 5 ;
    int32_t Ap [ ] = {0, 2, 5, 9, 10, 12} ;
    int32_t Ai [ ] = { 0,  1,  0,   2,  4,  1,  2,  3,   4,  2,  1,  4} ;
    double Ax [ ] = {2., 3., 3., -1., 4., 4., -3., 1., 2., 2., 6., 1.} ;
    double b [ ] = {8., 45., -3., 3., 19.} ;
    double x [5] ;

    int main (void)
    {
        double *null = (double *) NULL ;
        int i ;
        void *Symbolic, *Numeric ;
        (void) umfpack_di_symbolic (n, n, Ap, Ai, Ax, &Symbolic, null, null) ;
        (void) umfpack_di_numeric (Ap, Ai, Ax, Symbolic, &Numeric, null, null) ;
        umfpack_di_free_symbolic (&Symbolic) ;
        (void) umfpack_di_solve (UMFPACK_A, Ap, Ai, Ax, x, b, Numeric, null, null) ;
        umfpack_di_free_numeric (&Numeric) ;
        for (i = 0 ; i < n ; i++) printf ("x [%d] = %g\n", i, x [i]) ;
        return (0) ;
    }
\end{verbatim}
}

The {\tt Ap}, {\tt Ai}, and {\tt Ax} arrays represent the matrix
\[
\m{A} = \left[
\begin{array}{rrrrr}
 2 &  3 &  0 &  0 &  0 \\
 3 &  0 &  4 &  0 &  6 \\
 0 & -1 & -3 &  2 &  0 \\
 0 &  0 &  1 &  0 &  0 \\
 0 &  4 &  2 &  0 &  1 \\
\end{array}
\right].
\]
and the solution is $\m{x} = [1 \, 2 \, 3 \, 4 \, 5]\tr$.  The program uses
default control settings and does not return any statistics about the ordering,
factorization, or solution ({\tt Control} and {\tt Info} are both
{\tt (double *) NULL}).

For routines to manipulate a simpler ``triplet-form'' data structure for your
sparse matrix $\m{A}$, refer to the UMFPACK User Guide.

%-------------------------------------------------------------------------------
\section{Synopsis of primary C-callable routines}
\label{Synopsis}
%-------------------------------------------------------------------------------

The matrix $\m{A}$ is {\tt m}-by-{\tt n} with {\tt nz} entries.
The optional {\tt umfpack\_di\_defaults} routine loads the default control
parameters into the {\tt Control} array.  The settings can then be modified
before passing the array to the other routines.  Refer to the description
of each function in \verb'umfpack.h'.

{\footnotesize
\begin{verbatim}
    #include "umfpack.h"
    int status, sys ; int32_t n, m, nz, Ap [n+1], Ai [nz] ;
    double Control [UMFPACK_CONTROL], Info [UMFPACK_INFO], Ax [nz], X [n], B [n] ;
    void *Symbolic, *Numeric ;

    umfpack_di_defaults (Control) ;
    status = umfpack_di_symbolic (m, n, Ap, Ai, Ax, &Symbolic, Control, Info) ;
    status = umfpack_di_numeric (Ap, Ai, Ax, Symbolic, &Numeric, Control, Info) ;
    status = umfpack_di_solve (sys, Ap, Ai, Ax, X, B, Numeric, Control, Info) ;
    umfpack_di_free_symbolic (&Symbolic) ;
    umfpack_di_free_numeric (&Numeric) ;
\end{verbatim}
}

%-------------------------------------------------------------------------------
\section{Installation}
\label{Install}
%-------------------------------------------------------------------------------

You will need to install both UMFPACK and AMD to use UMFPACK.
The {\tt UMFPACK} and {\tt AMD} subdirectories must be placed side-by-side
within the same parent directory.  AMD is a stand-alone package that
is required by UMFPACK.  UMFPACK can be compiled without the
BLAS
but your performance will be much less than what it should be.

UMFPACK can optionally use CHOLMOD, CCAMD, CCOLAMD, COLAMD, and
\verb'SuiteSparse_metis' (a slightly modified version of the original METIS
v5.1.0) by default.  You can remove this dependency by compiling with the cmake
variable \verb'UMFPACK_USE_CHOLMOD' set to \verb'OFF'; see the
\verb'CMakeLists.txt' file.

CMake is used to build the UMFPACK library.  An optional top-level Makefile
simplifies its use.  To compile and install the library for system-wide usage:

\begin{verbatim}
        make ; sudo make install
\end{verbatim}

    To compile/install for local usage (SuiteSparse/lib and SuiteSparse/include)

\begin{verbatim}
        make local ; sudo make install
\end{verbatim}

    To run the demos

\begin{verbatim}
        make demos
\end{verbatim}

For Windows, simply import the \verb'CMakeLists.txt' script into Visual Studio.

Use the MATLAB command {\tt umfpack\_make} in the MATLAB directory
to compile UMFPACK and AMD for use in MATLAB.

The {\tt UMFPACK\_CONFIG} string can include combinations of the following;
most deal with how the BLAS are called:
\begin{itemize}
\item {\tt -DNBLAS} if you do not have any BLAS at all.
\item {\tt -DLONGBLAS} if your BLAS takes non-\verb'int32_t' integer arguments.
\item {\tt -DBLAS\_INT = } the integer used by the BLAS.

\item {\tt -DNRECIPROCAL} controls a trade-off between speed and accuracy.
    This is off by default (speed preferred over accuracy) except when
    compiling for MATLAB.
\end{itemize}

When you compile your program that uses the C-callable UMFPACK library,
you need to link your program with all libraries:
-lumfpack -lamd -lcholmod -lcolamd -lccolamd -lcamd -lmetis -lsuitesparseconfig.
If you don't compile UMFPACK to use METIS, then you can  just use
-lumfpack -lamd -lsuitesparseconfig.

All libraries are placed in {\tt SuiteSparse/lib} and all include files are
placed in {\tt SuiteSparse/include}, and \verb'make install' will also place
them where they are available system-wide.  To install for just yourself, use
\verb'make local' and then \verb'make install'.

You do not need to directly include any AMD include files in your
program, unless you directly call AMD routines.  You only need the
\begin{verbatim}
#include "umfpack.h"
\end{verbatim}
statement, as described in Section~\ref{Synopsis}.

%-------------------------------------------------------------------------------
\newpage
\section{The primary UMFPACK routines}
\label{Primary}
%-------------------------------------------------------------------------------

\subsection{umfpack\_di\_symbolic}

% INCLUDE umfpack_di_symbolic
{\footnotesize
\begin{verbatim}

int umfpack_di_symbolic
(
    int32_t n_row,
    int32_t n_col,
    const int32_t Ap [ ],
    const int32_t Ai [ ],
    const double Ax [ ],
    void **Symbolic,
    const double Control [UMFPACK_CONTROL],
    double Info [UMFPACK_INFO]
) ;

int umfpack_dl_symbolic
(
    int64_t n_row,
    int64_t n_col,
    const int64_t Ap [ ],
    const int64_t Ai [ ],
    const double Ax [ ],
    void **Symbolic,
    const double Control [UMFPACK_CONTROL],
    double Info [UMFPACK_INFO]
) ;

int umfpack_zi_symbolic
(
    int32_t n_row,
    int32_t n_col,
    const int32_t Ap [ ],
    const int32_t Ai [ ],
    const double Ax [ ], const double Az [ ],
    void **Symbolic,
    const double Control [UMFPACK_CONTROL],
    double Info [UMFPACK_INFO]
) ;

int umfpack_zl_symbolic
(
    int64_t n_row,
    int64_t n_col,
    const int64_t Ap [ ],
    const int64_t Ai [ ],
    const double Ax [ ], const double Az [ ],
    void **Symbolic,
    const double Control [UMFPACK_CONTROL],
    double Info [UMFPACK_INFO]
) ;

/*
double int32_t Syntax:

    #include "umfpack.h"
    void *Symbolic ;
    int32_t n_row, n_col, *Ap, *Ai ;
    double Control [UMFPACK_CONTROL], Info [UMFPACK_INFO], *Ax ;
    int status = umfpack_di_symbolic (n_row, n_col, Ap, Ai, Ax,
        &Symbolic, Control, Info) ;

double int64_t Syntax:

    #include "umfpack.h"
    void *Symbolic ;
    int64_t n_row, n_col, *Ap, *Ai ;
    double Control [UMFPACK_CONTROL], Info [UMFPACK_INFO], *Ax ;
    int status = umfpack_dl_symbolic (n_row, n_col, Ap, Ai, Ax,
        &Symbolic, Control, Info) ;

complex int32_t Syntax:

    #include "umfpack.h"
    void *Symbolic ;
    int32_t n_row, n_col, *Ap, *Ai ;
    double Control [UMFPACK_CONTROL], Info [UMFPACK_INFO], *Ax, *Az ;
    int status = umfpack_zi_symbolic (n_row, n_col, Ap, Ai, Ax, Az,
        &Symbolic, Control, Info) ;

complex int64_t Syntax:

    #include "umfpack.h"
    void *Symbolic ;
    int64_t n_row, n_col, *Ap, *Ai ;
    double Control [UMFPACK_CONTROL], Info [UMFPACK_INFO], *Ax, *Az ;
    int status = umfpack_zl_symbolic (n_row, n_col, Ap, Ai, Ax, Az,
        &Symbolic, Control, Info) ;

packed complex Syntax:

    Same as above, except Az is NULL.

Purpose:

    Given nonzero pattern of a sparse matrix A in column-oriented form,
    umfpack_*_symbolic performs a column pre-ordering to reduce fill-in
    (using COLAMD, AMD or METIS) and a symbolic factorization.  This is required
    before the matrix can be numerically factorized with umfpack_*_numeric.
    If you wish to bypass the COLAMD/AMD/METIS pre-ordering and provide your own
    ordering, use umfpack_*_qsymbolic instead.  If you wish to pass in a
    pointer to a user-provided ordering function, use umfpack_*_fsymbolic.

    Since umfpack_*_symbolic and umfpack_*_qsymbolic are very similar, options
    for both routines are discussed below.

    For the following discussion, let S be the submatrix of A obtained after
    eliminating all pivots of zero Markowitz cost.  S has dimension
    (n_row-n1-nempty_row) -by- (n_col-n1-nempty_col), where
    n1 = Info [UMFPACK_COL_SINGLETONS] + Info [UMFPACK_ROW_SINGLETONS],
    nempty_row = Info [UMFPACK_NEMPTY_ROW] and
    nempty_col = Info [UMFPACK_NEMPTY_COL].

Returns:

    The status code is returned.  See Info [UMFPACK_STATUS], below.

Arguments:

    Int n_row ;         Input argument, not modified.
    Int n_col ;         Input argument, not modified.

        A is an n_row-by-n_col matrix.  Restriction: n_row > 0 and n_col > 0.

    Int Ap [n_col+1] ;  Input argument, not modified.

        Ap is an integer array of size n_col+1.  On input, it holds the
        "pointers" for the column form of the sparse matrix A.  Column j of
        the matrix A is held in Ai [(Ap [j]) ... (Ap [j+1]-1)].  The first
        entry, Ap [0], must be zero, and Ap [j] <= Ap [j+1] must hold for all
        j in the range 0 to n_col-1.  The value nz = Ap [n_col] is thus the
        total number of entries in the pattern of the matrix A.  nz must be
        greater than or equal to zero.

    Int Ai [nz] ;       Input argument, not modified, of size nz = Ap [n_col].

        The nonzero pattern (row indices) for column j is stored in
        Ai [(Ap [j]) ... (Ap [j+1]-1)].  The row indices in a given column j
        must be in ascending order, and no duplicate row indices may be present.
        Row indices must be in the range 0 to n_row-1 (the matrix is 0-based).
        See umfpack_*_triplet_to_col for how to sort the columns of a matrix
        and sum up the duplicate entries.  See umfpack_*_report_matrix for how
        to print the matrix A.

    double Ax [nz] ;    Optional input argument, not modified.  May be NULL.
                        Size 2*nz for packed complex case.

        The numerical values of the sparse matrix A.  The nonzero pattern (row
        indices) for column j is stored in Ai [(Ap [j]) ... (Ap [j+1]-1)], and
        the corresponding numerical values are stored in
        Ax [(Ap [j]) ... (Ap [j+1]-1)].  Used only for gathering statistics
        about how many nonzeros are placed on the diagonal by the fill-reducing
        ordering.

    double Az [nz] ;    Optional input argument, not modified, for complex
                        versions.  May be NULL.

        For the complex versions, this holds the imaginary part of A.  The
        imaginary part of column j is held in Az [(Ap [j]) ... (Ap [j+1]-1)].

        If Az is NULL, then both real
        and imaginary parts are contained in Ax[0..2*nz-1], with Ax[2*k]
        and Ax[2*k+1] being the real and imaginary part of the kth entry.

        Used for statistics only.  See the description of Ax, above.

    void **Symbolic ;   Output argument.

        **Symbolic is the address of a (void *) pointer variable in the user's
        calling routine (see Syntax, above).  On input, the contents of this
        variable are not defined.  On output, this variable holds a (void *)
        pointer to the Symbolic object (if successful), or (void *) NULL if
        a failure occurred.

    double Control [UMFPACK_CONTROL] ;  Input argument, not modified.

        If a (double *) NULL pointer is passed, then the default control
        settings are used (the defaults are suitable for all matrices,
        ranging from those with highly unsymmetric nonzero pattern, to
        symmetric matrices).  Otherwise, the settings are determined from the
        Control array.  See umfpack_*_defaults on how to fill the Control
        array with the default settings.  If Control contains NaN's, the
        defaults are used.  The following Control parameters are used:

        Control [UMFPACK_STRATEGY]:  This is the most important control
            parameter.  It determines what kind of ordering and pivoting
            strategy that UMFPACK should use.

            NOTE: the interaction of numerical and fill-reducing pivoting is
            a delicate balance, and a perfect hueristic is not possible because
            sparsity-preserving pivoting is an NP-hard problem.  Selecting the
            wrong strategy can lead to catastrophic fill-in and/or numerical
            inaccuracy.

            UMFPACK_STRATEGY_AUTO:  This is the default.  The input matrix is
                analyzed to determine how symmetric the nonzero pattern is, and
                how many entries there are on the diagonal.  It then selects one
                of the following strategies.  Refer to the User Guide for a
                description of how the strategy is automatically selected.

            UMFPACK_STRATEGY_UNSYMMETRIC:  Use the unsymmetric strategy.  COLAMD
                is used to order the columns of A, followed by a postorder of
                the column elimination tree.  No attempt is made to perform
                diagonal pivoting.  The column ordering is refined during
                factorization.

                In the numerical factorization, the
                Control [UMFPACK_SYM_PIVOT_TOLERANCE] parameter is ignored.  A
                pivot is selected if its magnitude is >=
                Control [UMFPACK_PIVOT_TOLERANCE] (default 0.1) times the
                largest entry in its column.

            UMFPACK_STRATEGY_SYMMETRIC:  Use the symmetric strategy
                In this method, the approximate minimum degree
                ordering (AMD) is applied to A+A', followed by a postorder of
                the elimination tree of A+A'.  UMFPACK attempts to perform
                diagonal pivoting during numerical factorization.  No refinement
                of the column pre-ordering is performed during factorization.

                In the numerical factorization, a nonzero entry on the diagonal
                is selected as the pivot if its magnitude is >= Control
                [UMFPACK_SYM_PIVOT_TOLERANCE] (default 0.001) times the largest
                entry in its column.  If this is not acceptable, then an
                off-diagonal pivot is selected with magnitude >= Control
                [UMFPACK_PIVOT_TOLERANCE] (default 0.1) times the largest entry
                in its column.

        Control [UMFPACK_ORDERING]:  The ordering method to use:
            UMFPACK_ORDERING_CHOLMOD    try AMD/COLAMD, then METIS if needed
            UMFPACK_ORDERING_AMD        just AMD or COLAMD
            UMFPACK_ORDERING_GIVEN      just Qinit (umfpack_*_qsymbolic only)
            UMFPACK_ORDERING_NONE       no fill-reducing ordering
            UMFPACK_ORDERING_METIS      just METIS(A+A') or METIS(A'A)
            UMFPACK_ORDERING_BEST       try AMD/COLAMD, METIS, and NESDIS
            UMFPACK_ORDERING_USER       just user function (*_fsymbolic only)
            UMFPACK_ORDERING_METIS_GUARD use METIS, AMD, or COLAMD.
                Symmetric strategy: always use METIS on A+A'.  Unsymmetric
                strategy: use METIS on A'A, unless A has one or more very dense
                rows.  In that case, A'A is very costly to form.  In this case,
                COLAMD is used instead of METIS.

        Control [UMFPACK_SINGLETONS]: If false (0), then singletons are
            not removed prior to factorization.  Default: true (1).

        Control [UMFPACK_DENSE_COL]:
            If COLAMD is used, columns with more than
            max (16, Control [UMFPACK_DENSE_COL] * 16 * sqrt (n_row)) entries
            are placed placed last in the column pre-ordering.  Default: 0.2.

        Control [UMFPACK_DENSE_ROW]:
            Rows with more than max (16, Control [UMFPACK_DENSE_ROW] * 16 *
            sqrt (n_col)) entries are treated differently in the COLAMD
            pre-ordering, and in the internal data structures during the
            subsequent numeric factorization.  Default: 0.2.
            If any row exists with more than these number of entries, and
            if the unsymmetric strategy is selected, the METIS_GUARD ordering
            selects COLAMD instead of METIS.

        Control [UMFPACK_AMD_DENSE]:  rows/columns in A+A' with more than
            max (16, Control [UMFPACK_AMD_DENSE] * sqrt (n)) entries
            (where n = n_row = n_col) are ignored in the AMD pre-ordering.
            Default: 10.

        Control [UMFPACK_BLOCK_SIZE]:  the block size to use for Level-3 BLAS
            in the subsequent numerical factorization (umfpack_*_numeric).
            A value less than 1 is treated as 1.  Default: 32.  Modifying this
            parameter affects when updates are applied to the working frontal
            matrix, and can indirectly affect fill-in and operation count.
            Assuming the block size is large enough (8 or so), this parameter
            has a modest effect on performance.

        Control [UMFPACK_FIXQ]:  If > 0, then the pre-ordering Q is not modified
            during numeric factorization.  If < 0, then Q may be modified.  If
            zero, then this is controlled automatically (the unsymmetric
            strategy modifies Q, the others do not).  Default: 0.

            Note that the symbolic analysis will in general modify the input
            ordering Qinit to obtain Q; see umfpack_qsymbolic.h for details.
            This option ensures Q does not change, as found in the symbolic
            analysis, but Qinit is in general not the same as Q.

        Control [UMFPACK_AGGRESSIVE]:  If nonzero, aggressive absorption is used
            in COLAMD and AMD.  Default: 1.

        // added for v6.0.0:
        Control [UMFPACK_STRATEGY_THRESH_SYM]: tsym, Default 0.5.
        Control [UMFPACK_STRATEGY_THRESH_NNZDIAG]: tdiag, Default 0.9.
            For the auto strategy, if the pattern of the submatrix S after
            removing singletons has a symmetry of tsym or more (0 being
            completely unsymmetric and 1 being completely symmetric, and if the
            fraction of entries present on the diagonal is >= tdiag, then the
            symmetric strategy is chosen.  Otherwise, the unsymmetric strategy
            is chosen.

    double Info [UMFPACK_INFO] ;        Output argument, not defined on input.

        Contains statistics about the symbolic analysis.  If a (double *) NULL
        pointer is passed, then no statistics are returned in Info (this is not
        an error condition).  The entire Info array is cleared (all entries set
        to -1) and then the following statistics are computed:

        Info [UMFPACK_STATUS]: status code.  This is also the return value,
            whether or not Info is present.

            UMFPACK_OK

                Each column of the input matrix contained row indices
                in increasing order, with no duplicates.  Only in this case
                does umfpack_*_symbolic compute a valid symbolic factorization.
                For the other cases below, no Symbolic object is created
                (*Symbolic is (void *) NULL).

            UMFPACK_ERROR_n_nonpositive

                n is less than or equal to zero.

            UMFPACK_ERROR_invalid_matrix

                Number of entries in the matrix is negative, Ap [0] is nonzero,
                a column has a negative number of entries, a row index is out of
                bounds, or the columns of input matrix were jumbled (unsorted
                columns or duplicate entries).

            UMFPACK_ERROR_out_of_memory

                Insufficient memory to perform the symbolic analysis.  If the
                analysis requires more than 2GB of memory and you are using
                the int32_t version of UMFPACK, then you are guaranteed
                to run out of memory.  Try using the 64-bit version of UMFPACK.

            UMFPACK_ERROR_argument_missing

                One or more required arguments is missing.

            UMFPACK_ERROR_internal_error

                Something very serious went wrong.  This is a bug.
                Please contact the author (DrTimothyAldenDavis@gmail.com).

        Info [UMFPACK_NROW]:  the value of the input argument n_row.

        Info [UMFPACK_NCOL]:  the value of the input argument n_col.

        Info [UMFPACK_NZ]:  the number of entries in the input matrix
            (Ap [n_col]).

        Info [UMFPACK_SIZE_OF_UNIT]:  the number of bytes in a Unit,
            for memory usage statistics below.

        Info [UMFPACK_SIZE_OF_INT]:  the number of bytes in an int32_t.

        Info [UMFPACK_SIZE_OF_LONG]:  the number of bytes in a int64_t.

        Info [UMFPACK_SIZE_OF_POINTER]:  the number of bytes in a void *
            pointer.

        Info [UMFPACK_SIZE_OF_ENTRY]:  the number of bytes in a numerical entry.

        Info [UMFPACK_NDENSE_ROW]:  number of "dense" rows in A.  These rows are
            ignored when the column pre-ordering is computed in COLAMD.  They
            are also treated differently during numeric factorization.  If > 0,
            then the matrix had to be re-analyzed by UMF_analyze, which does
            not ignore these rows.

        Info [UMFPACK_NEMPTY_ROW]:  number of "empty" rows in A, as determined
            These are rows that either have no entries, or whose entries are
            all in pivot columns of zero-Markowitz-cost pivots.

        Info [UMFPACK_NDENSE_COL]:  number of "dense" columns in A.  COLAMD
            orders these columns are ordered last in the factorization, but
            before "empty" columns.

        Info [UMFPACK_NEMPTY_COL]:  number of "empty" columns in A.  These are
            columns that either have no entries, or whose entries are all in
            pivot rows of zero-Markowitz-cost pivots.  These columns are
            ordered last in the factorization, to the right of "dense" columns.

        Info [UMFPACK_SYMBOLIC_DEFRAG]:  number of garbage collections
            performed during ordering and symbolic pre-analysis.

        Info [UMFPACK_SYMBOLIC_PEAK_MEMORY]:  the amount of memory (in Units)
            required for umfpack_*_symbolic to complete.  This count includes
            the size of the Symbolic object itself, which is also reported in
            Info [UMFPACK_SYMBOLIC_SIZE].

        Info [UMFPACK_SYMBOLIC_SIZE]: the final size of the Symbolic object (in
            Units).  This is fairly small, roughly 2*n to 13*n integers,
            depending on the matrix.

        Info [UMFPACK_VARIABLE_INIT_ESTIMATE]: the Numeric object contains two
            parts.  The first is fixed in size (O (n_row+n_col)).  The
            second part holds the sparse LU factors and the contribution blocks
            from factorized frontal matrices.  This part changes in size during
            factorization.  Info [UMFPACK_VARIABLE_INIT_ESTIMATE] is the exact
            size (in Units) required for this second variable-sized part in
            order for the numerical factorization to start.

        Info [UMFPACK_VARIABLE_PEAK_ESTIMATE]: the estimated peak size (in
            Units) of the variable-sized part of the Numeric object.  This is
            usually an upper bound, but that is not guaranteed.

        Info [UMFPACK_VARIABLE_FINAL_ESTIMATE]: the estimated final size (in
            Units) of the variable-sized part of the Numeric object.  This is
            usually an upper bound, but that is not guaranteed.  It holds just
            the sparse LU factors.

        Info [UMFPACK_NUMERIC_SIZE_ESTIMATE]:  an estimate of the final size (in
            Units) of the entire Numeric object (both fixed-size and variable-
            sized parts), which holds the LU factorization (including the L, U,
            P and Q matrices).

        Info [UMFPACK_PEAK_MEMORY_ESTIMATE]:  an estimate of the total amount of
            memory (in Units) required by umfpack_*_symbolic and
            umfpack_*_numeric to perform both the symbolic and numeric
            factorization.  This is the larger of the amount of memory needed
            in umfpack_*_numeric itself, and the amount of memory needed in
            umfpack_*_symbolic (Info [UMFPACK_SYMBOLIC_PEAK_MEMORY]).  The
            count includes the size of both the Symbolic and Numeric objects
            themselves.  It can be a very loose upper bound, particularly when
            the symmetric strategy is used.

        Info [UMFPACK_FLOPS_ESTIMATE]:  an estimate of the total floating-point
            operations required to factorize the matrix.  This is a "true"
            theoretical estimate of the number of flops that would be performed
            by a flop-parsimonious sparse LU algorithm.  It assumes that no
            extra flops are performed except for what is strictly required to
            compute the LU factorization.  It ignores, for example, the flops
            performed by umfpack_di_numeric to add contribution blocks of
            frontal matrices together.  If L and U are the upper bound on the
            pattern of the factors, then this flop count estimate can be
            represented in MATLAB (for real matrices, not complex) as:

                Lnz = full (sum (spones (L))) - 1 ;     % nz in each col of L
                Unz = full (sum (spones (U')))' - 1 ;   % nz in each row of U
                flops = 2*Lnz*Unz + sum (Lnz) ;

            The actual "true flop" count found by umfpack_*_numeric will be
            less than this estimate.

            For the real version, only (+ - * /) are counted.  For the complex
            version, the following counts are used:

                operation       flops
                c = 1/b         6
                c = a*b         6
                c -= a*b        8

        Info [UMFPACK_LNZ_ESTIMATE]:  an estimate of the number of nonzeros in
            L, including the diagonal.  Since L is unit-diagonal, the diagonal
            of L is not stored.  This estimate is a strict upper bound on the
            actual nonzeros in L to be computed by umfpack_*_numeric.

        Info [UMFPACK_UNZ_ESTIMATE]:  an estimate of the number of nonzeros in
            U, including the diagonal.  This estimate is a strict upper bound on
            the actual nonzeros in U to be computed by umfpack_*_numeric.

        Info [UMFPACK_MAX_FRONT_SIZE_ESTIMATE]: estimate of the size of the
            largest frontal matrix (# of entries), for arbitrary partial
            pivoting during numerical factorization.

        Info [UMFPACK_SYMBOLIC_TIME]:  The CPU time taken, in seconds.

        Info [UMFPACK_SYMBOLIC_WALLTIME]:  The wallclock time taken, in seconds.

        Info [UMFPACK_STRATEGY_USED]: The ordering strategy used:
            UMFPACK_STRATEGY_SYMMETRIC or UMFPACK_STRATEGY_UNSYMMETRIC

        Info [UMFPACK_ORDERING_USED]:  The ordering method used:
            UMFPACK_ORDERING_AMD    (AMD for sym. strategy, COLAMD for unsym.)
            UMFPACK_ORDERING_GIVEN
            UMFPACK_ORDERING_NONE
            UMFPACK_ORDERING_METIS
            UMFPACK_ORDERING_USER

        Info [UMFPACK_QFIXED]: 1 if the column pre-ordering will be refined
            during numerical factorization, 0 if not.

        Info [UMFPACK_DIAG_PREFERED]: 1 if diagonal pivoting will be attempted,
            0 if not.

        Info [UMFPACK_COL_SINGLETONS]:  the matrix A is analyzed by first
            eliminating all pivots with zero Markowitz cost.  This count is the
            number of these pivots with exactly one nonzero in their pivot
            column.

        Info [UMFPACK_ROW_SINGLETONS]:  the number of zero-Markowitz-cost
            pivots with exactly one nonzero in their pivot row.

        Info [UMFPACK_PATTERN_SYMMETRY]: the symmetry of the pattern of S.

        Info [UMFPACK_NZ_A_PLUS_AT]: the number of off-diagonal entries in S+S'.

        Info [UMFPACK_NZDIAG]:  the number of entries on the diagonal of S.

        Info [UMFPACK_N2]:  if S is square, and nempty_row = nempty_col, this
            is equal to n_row - n1 - nempty_row.

        Info [UMFPACK_S_SYMMETRIC]: 1 if S is square and its diagonal has been
            preserved, 0 otherwise.


        Info [UMFPACK_MAX_FRONT_NROWS_ESTIMATE]: estimate of the max number of
            rows in any frontal matrix, for arbitrary partial pivoting.

        Info [UMFPACK_MAX_FRONT_NCOLS_ESTIMATE]: estimate of the max number of
            columns in any frontal matrix, for arbitrary partial pivoting.

        ------------------------------------------------------------------------
        The next four statistics are computed only if AMD is used:
        ------------------------------------------------------------------------

        Info [UMFPACK_SYMMETRIC_LUNZ]: The number of nonzeros in L and U,
            assuming no pivoting during numerical factorization, and assuming a
            zero-free diagonal of U.  Excludes the entries on the diagonal of
            L.  If the matrix has a purely symmetric nonzero pattern, this is
            often a lower bound on the nonzeros in the actual L and U computed
            in the numerical factorization, for matrices that fit the criteria
            for the "symmetric" strategy.

        Info [UMFPACK_SYMMETRIC_FLOPS]: The floating-point operation count in
            the numerical factorization phase, assuming no pivoting.  If the
            pattern of the matrix is symmetric, this is normally a lower bound
            on the floating-point operation count in the actual numerical
            factorization, for matrices that fit the criteria for the symmetric
            strategy.

        Info [UMFPACK_SYMMETRIC_NDENSE]: The number of "dense" rows/columns of
            S+S' that were ignored during the AMD ordering.  These are placed
            last in the output order.  If > 0, then the
            Info [UMFPACK_SYMMETRIC_*] statistics, above are rough upper bounds.

        Info [UMFPACK_SYMMETRIC_DMAX]: The maximum number of nonzeros in any
            column of L, if no pivoting is performed during numerical
            factorization.  Excludes the part of the LU factorization for
            pivots with zero Markowitz cost.

        At the start of umfpack_*_symbolic, all of Info is set of -1, and then
        after that only the above listed Info [...] entries are accessed.
        Future versions might modify different parts of Info.
*/

\end{verbatim}
}


%-------------------------------------------------------------------------------
\newpage
\subsection{umfpack\_di\_numeric}

% INCLUDE umfpack_di_numeric
{\footnotesize
\begin{verbatim}

int umfpack_di_numeric
(
    const int32_t Ap [ ],
    const int32_t Ai [ ],
    const double Ax [ ],
    void *Symbolic,
    void **Numeric,
    const double Control [UMFPACK_CONTROL],
    double Info [UMFPACK_INFO]
) ;

int umfpack_dl_numeric
(
    const int64_t Ap [ ],
    const int64_t Ai [ ],
    const double Ax [ ],
    void *Symbolic,
    void **Numeric,
    const double Control [UMFPACK_CONTROL],
    double Info [UMFPACK_INFO]
) ;

int umfpack_zi_numeric
(
    const int32_t Ap [ ],
    const int32_t Ai [ ],
    const double Ax [ ], const double Az [ ],
    void *Symbolic,
    void **Numeric,
    const double Control [UMFPACK_CONTROL],
    double Info [UMFPACK_INFO]
) ;

int umfpack_zl_numeric
(
    const int64_t Ap [ ],
    const int64_t Ai [ ],
    const double Ax [ ], const double Az [ ],
    void *Symbolic,
    void **Numeric,
    const double Control [UMFPACK_CONTROL],
    double Info [UMFPACK_INFO]
) ;

/*
double int32_t Syntax:

    #include "umfpack.h"
    void *Symbolic, *Numeric ;
    int32_t *Ap, *Ai, status ;
    double *Ax, Control [UMFPACK_CONTROL], Info [UMFPACK_INFO] ;
    int status = umfpack_di_numeric (Ap, Ai, Ax, Symbolic, &Numeric, Control,
        Info) ;

double int64_t Syntax:

    #include "umfpack.h"
    void *Symbolic, *Numeric ;
    int64_t *Ap, *Ai ;
    double *Ax, Control [UMFPACK_CONTROL], Info [UMFPACK_INFO] ;
    int status = umfpack_dl_numeric (Ap, Ai, Ax, Symbolic, &Numeric, Control,
        Info) ;

complex int32_t Syntax:

    #include "umfpack.h"
    void *Symbolic, *Numeric ;
    int32_t *Ap, *Ai ;
    double *Ax, *Az, Control [UMFPACK_CONTROL], Info [UMFPACK_INFO] ;
    int status = umfpack_zi_numeric (Ap, Ai, Ax, Az, Symbolic, &Numeric,
        Control, Info) ;

complex int64_t Syntax:

    #include "umfpack.h"
    void *Symbolic, *Numeric ;
    int64_t *Ap, *Ai ;
    double *Ax, *Az, Control [UMFPACK_CONTROL], Info [UMFPACK_INFO] ;
    int status = umfpack_zl_numeric (Ap, Ai, Ax, Az, Symbolic, &Numeric,
        Control, Info) ;

packed complex Syntax:

    Same as above, except that Az is NULL.

Purpose:

    Given a sparse matrix A in column-oriented form, and a symbolic analysis
    computed by umfpack_*_*symbolic, the umfpack_*_numeric routine performs the
    numerical factorization, PAQ=LU, PRAQ=LU, or P(R\A)Q=LU, where P and Q are
    permutation matrices (represented as permutation vectors), R is the row
    scaling, L is unit-lower triangular, and U is upper triangular.  This is
    required before the system Ax=b (or other related linear systems) can be
    solved.  umfpack_*_numeric can be called multiple times for each call to
    umfpack_*_*symbolic, to factorize a sequence of matrices with identical
    nonzero pattern.  Simply compute the Symbolic object once, with
    umfpack_*_*symbolic, and reuse it for subsequent matrices.  This routine
    safely detects if the pattern changes, and sets an appropriate error code.

Returns:

    The status code is returned.  See Info [UMFPACK_STATUS], below.

Arguments:

    Int Ap [n_col+1] ;  Input argument, not modified.

        This must be identical to the Ap array passed to umfpack_*_*symbolic.
        The value of n_col is what was passed to umfpack_*_*symbolic (this is
        held in the Symbolic object).

    Int Ai [nz] ;       Input argument, not modified, of size nz = Ap [n_col].

        This must be identical to the Ai array passed to umfpack_*_*symbolic.

    double Ax [nz] ;    Input argument, not modified, of size nz = Ap [n_col].
                        Size 2*nz for packed complex case.

        The numerical values of the sparse matrix A.  The nonzero pattern (row
        indices) for column j is stored in Ai [(Ap [j]) ... (Ap [j+1]-1)], and
        the corresponding numerical values are stored in
        Ax [(Ap [j]) ... (Ap [j+1]-1)].

    double Az [nz] ;    Input argument, not modified, for complex versions.

        For the complex versions, this holds the imaginary part of A.  The
        imaginary part of column j is held in Az [(Ap [j]) ... (Ap [j+1]-1)].

        If Az is NULL, then both real
        and imaginary parts are contained in Ax[0..2*nz-1], with Ax[2*k]
        and Ax[2*k+1] being the real and imaginary part of the kth entry.

    void *Symbolic ;    Input argument, not modified.

        The Symbolic object, which holds the symbolic factorization computed by
        umfpack_*_*symbolic.  The Symbolic object is not modified by
        umfpack_*_numeric.

    void **Numeric ;    Output argument.

        **Numeric is the address of a (void *) pointer variable in the user's
        calling routine (see Syntax, above).  On input, the contents of this
        variable are not defined.  On output, this variable holds a (void *)
        pointer to the Numeric object (if successful), or (void *) NULL if
        a failure occurred.

    double Control [UMFPACK_CONTROL] ;   Input argument, not modified.

        If a (double *) NULL pointer is passed, then the default control
        settings are used.  Otherwise, the settings are determined from the
        Control array.  See umfpack_*_defaults on how to fill the Control
        array with the default settings.  If Control contains NaN's, the
        defaults are used.  The following Control parameters are used:

        Control [UMFPACK_PIVOT_TOLERANCE]:  relative pivot tolerance for
            threshold partial pivoting with row interchanges.  In any given
            column, an entry is numerically acceptable if its absolute value is
            greater than or equal to Control [UMFPACK_PIVOT_TOLERANCE] times
            the largest absolute value in the column.  A value of 1.0 gives true
            partial pivoting.  If less than or equal to zero, then any nonzero
            entry is numerically acceptable as a pivot.  Default: 0.1.

            Smaller values tend to lead to sparser LU factors, but the solution
            to the linear system can become inaccurate.  Larger values can lead
            to a more accurate solution (but not always), and usually an
            increase in the total work.

            For complex matrices, a cheap approximate of the absolute value
            is used for the threshold partial pivoting test (|a_real| + |a_imag|
            instead of the more expensive-to-compute exact absolute value
            sqrt (a_real^2 + a_imag^2)).

        Control [UMFPACK_SYM_PIVOT_TOLERANCE]:
            If diagonal pivoting is attempted (the symmetric
            strategy is used) then this parameter is used to control when the
            diagonal entry is selected in a given pivot column.  The absolute
            value of the entry must be >= Control [UMFPACK_SYM_PIVOT_TOLERANCE]
            times the largest absolute value in the column.  A value of zero
            will ensure that no off-diagonal pivoting is performed, except that
            zero diagonal entries are not selected if there are any off-diagonal
            nonzero entries.

            If an off-diagonal pivot is selected, an attempt is made to restore
            symmetry later on.  Suppose A (i,j) is selected, where i != j.
            If column i has not yet been selected as a pivot column, then
            the entry A (j,i) is redefined as a "diagonal" entry, except that
            the tighter tolerance (Control [UMFPACK_PIVOT_TOLERANCE]) is
            applied.  This strategy has an effect similar to 2-by-2 pivoting
            for symmetric indefinite matrices.  If a 2-by-2 block pivot with
            nonzero structure

                       i j
                    i: 0 x
                    j: x 0

            is selected in a symmetric indefinite factorization method, the
            2-by-2 block is inverted and a rank-2 update is applied.  In
            UMFPACK, this 2-by-2 block would be reordered as

                       j i
                    i: x 0
                    j: 0 x

            In both cases, the symmetry of the Schur complement is preserved.

        Control [UMFPACK_SCALE]:  Note that the user's input matrix is
            never modified, only an internal copy is scaled.

            There are three valid settings for this parameter.  If any other
            value is provided, the default is used.

            UMFPACK_SCALE_NONE:  no scaling is performed.

            UMFPACK_SCALE_SUM:  each row of the input matrix A is divided by
                the sum of the absolute values of the entries in that row.
                The scaled matrix has an infinity norm of 1.

            UMFPACK_SCALE_MAX:  each row of the input matrix A is divided by
                the maximum the absolute values of the entries in that row.
                In the scaled matrix the largest entry in each row has
                a magnitude exactly equal to 1.

            Note that for complex matrices, a cheap approximate absolute value
            is used, |a_real| + |a_imag|, instead of the exact absolute value
            sqrt ((a_real)^2 + (a_imag)^2).

            Scaling is very important for the "symmetric" strategy when
            diagonal pivoting is attempted.  It also improves the performance
            of the "unsymmetric" strategy.

            Default: UMFPACK_SCALE_SUM.

        Control [UMFPACK_ALLOC_INIT]:

            When umfpack_*_numeric starts, it allocates memory for the Numeric
            object.  Part of this is of fixed size (approximately n double's +
            12*n integers).  The remainder is of variable size, which grows to
            hold the LU factors and the frontal matrices created during
            factorization.  A estimate of the upper bound is computed by
            umfpack_*_*symbolic, and returned by umfpack_*_*symbolic in
            Info [UMFPACK_VARIABLE_PEAK_ESTIMATE] (in Units).

            If Control [UMFPACK_ALLOC_INIT] is >= 0, umfpack_*_numeric initially
            allocates space for the variable-sized part equal to this estimate
            times Control [UMFPACK_ALLOC_INIT].  Typically, for matrices for
            which the "unsymmetric" strategy applies, umfpack_*_numeric needs
            only about half the estimated memory space, so a setting of 0.5 or
            0.6 often provides enough memory for umfpack_*_numeric to factorize
            the matrix with no subsequent increases in the size of this block.

            If the matrix is ordered via AMD, then this non-negative parameter
            is ignored.  The initial allocation ratio computed automatically,
            as 1.2 * (nz + Info [UMFPACK_SYMMETRIC_LUNZ]) /
            (Info [UMFPACK_LNZ_ESTIMATE] + Info [UMFPACK_UNZ_ESTIMATE] -
            min (n_row, n_col)).

            If Control [UMFPACK_ALLOC_INIT] is negative, then umfpack_*_numeric
            allocates a space with initial size (in Units) equal to
            (-Control [UMFPACK_ALLOC_INIT]).

            Regardless of the value of this parameter, a space equal to or
            greater than the the bare minimum amount of memory needed to start
            the factorization is always initially allocated.  The bare initial
            memory required is returned by umfpack_*_*symbolic in
            Info [UMFPACK_VARIABLE_INIT_ESTIMATE] (an exact value, not an
            estimate).

            If the variable-size part of the Numeric object is found to be too
            small sometime after numerical factorization has started, the memory
            is increased in size by a factor of 1.2.   If this fails, the
            request is reduced by a factor of 0.95 until it succeeds, or until
            it determines that no increase in size is possible.  Garbage
            collection then occurs.

            The strategy of attempting to "malloc" a working space, and
            re-trying with a smaller space, may not work when UMFPACK is used
            as a mexFunction MATLAB, since mxMalloc aborts the mexFunction if it
            fails.  This issue does not affect the use of UMFPACK as a part of
            the built-in x=A\b in MATLAB 6.5 and later.

            If you are using the umfpack mexFunction, decrease the magnitude of
            Control [UMFPACK_ALLOC_INIT] if you run out of memory in MATLAB.

            Default initial allocation size: 0.7.  Thus, with the default
            control settings and the "unsymmetric" strategy, the upper-bound is
            reached after two reallocations (0.7 * 1.2 * 1.2 = 1.008).

            Changing this parameter has little effect on fill-in or operation
            count.  It has a small impact on run-time (the extra time required
            to do the garbage collection and memory reallocation).

        Control [UMFPACK_FRONT_ALLOC_INIT]:

            When UMFPACK starts the factorization of each "chain" of frontal
            matrices, it allocates a working array to hold the frontal matrices
            as they are factorized.  The symbolic factorization computes the
            size of the largest possible frontal matrix that could occur during
            the factorization of each chain.

            If Control [UMFPACK_FRONT_ALLOC_INIT] is >= 0, the following
            strategy is used.  If the AMD ordering was used, this non-negative
            parameter is ignored.  A front of size (d+2)*(d+2) is allocated,
            where d = Info [UMFPACK_SYMMETRIC_DMAX].  Otherwise, a front of
            size Control [UMFPACK_FRONT_ALLOC_INIT] times the largest front
            possible for this chain is allocated.

            If Control [UMFPACK_FRONT_ALLOC_INIT] is negative, then a front of
            size (-Control [UMFPACK_FRONT_ALLOC_INIT]) is allocated (where the
            size is in terms of the number of numerical entries).  This is done
            regardless of the ordering method or ordering strategy used.

            Default: 0.5.

        Control [UMFPACK_DROPTOL]:

            Entries in L and U with absolute value less than or equal to the
            drop tolerance are removed from the data structures (unless leaving
            them there reduces memory usage by reducing the space required
            for the nonzero pattern of L and U).

            Default: 0.0.

    double Info [UMFPACK_INFO] ;        Output argument.

        Contains statistics about the numeric factorization.  If a
        (double *) NULL pointer is passed, then no statistics are returned in
        Info (this is not an error condition).  The following statistics are
        computed in umfpack_*_numeric:

        Info [UMFPACK_STATUS]: status code.  This is also the return value,
            whether or not Info is present.

            UMFPACK_OK

                Numeric factorization was successful.  umfpack_*_numeric
                computed a valid numeric factorization.

            UMFPACK_WARNING_singular_matrix

                Numeric factorization was successful, but the matrix is
                singular.  umfpack_*_numeric computed a valid numeric
                factorization, but you will get a divide by zero in
                umfpack_*_*solve.  For the other cases below, no Numeric object
                is created (*Numeric is (void *) NULL).

            UMFPACK_ERROR_out_of_memory

                Insufficient memory to complete the numeric factorization.

            UMFPACK_ERROR_argument_missing

                One or more required arguments are missing.

            UMFPACK_ERROR_invalid_Symbolic_object

                Symbolic object provided as input is invalid.

            UMFPACK_ERROR_different_pattern

                The pattern (Ap and/or Ai) has changed since the call to
                umfpack_*_*symbolic which produced the Symbolic object.

        Info [UMFPACK_NROW]:  the value of n_row stored in the Symbolic object.

        Info [UMFPACK_NCOL]:  the value of n_col stored in the Symbolic object.

        Info [UMFPACK_NZ]:  the number of entries in the input matrix.
            This value is obtained from the Symbolic object.

        Info [UMFPACK_SIZE_OF_UNIT]:  the number of bytes in a Unit, for memory
            usage statistics below.

        Info [UMFPACK_VARIABLE_INIT]: the initial size (in Units) of the
            variable-sized part of the Numeric object.  If this differs from
            Info [UMFPACK_VARIABLE_INIT_ESTIMATE], then the pattern (Ap and/or
            Ai) has changed since the last call to umfpack_*_*symbolic, which is
            an error condition.

        Info [UMFPACK_VARIABLE_PEAK]: the peak size (in Units) of the
            variable-sized part of the Numeric object.  This size is the amount
            of space actually used inside the block of memory, not the space
            allocated via UMF_malloc.  You can reduce UMFPACK's memory
            requirements by setting Control [UMFPACK_ALLOC_INIT] to the ratio
            Info [UMFPACK_VARIABLE_PEAK] / Info[UMFPACK_VARIABLE_PEAK_ESTIMATE].
            This will ensure that no memory reallocations occur (you may want to
            add 0.001 to make sure that integer roundoff does not lead to a
            memory size that is 1 Unit too small; otherwise, garbage collection
            and reallocation will occur).

        Info [UMFPACK_VARIABLE_FINAL]: the final size (in Units) of the
            variable-sized part of the Numeric object.  It holds just the
            sparse LU factors.

        Info [UMFPACK_NUMERIC_SIZE]:  the actual final size (in Units) of the
            entire Numeric object, including the final size of the variable
            part of the object.  Info [UMFPACK_NUMERIC_SIZE_ESTIMATE],
            an estimate, was computed by umfpack_*_*symbolic.  The estimate is
            normally an upper bound on the actual final size, but this is not
            guaranteed.

        Info [UMFPACK_PEAK_MEMORY]:  the actual peak memory usage (in Units) of
            both umfpack_*_*symbolic and umfpack_*_numeric.  An estimate,
            Info [UMFPACK_PEAK_MEMORY_ESTIMATE], was computed by
            umfpack_*_*symbolic.  The estimate is normally an upper bound on the
            actual peak usage, but this is not guaranteed.  With testing on
            hundreds of matrix arising in real applications, I have never
            observed a matrix where this estimate or the Numeric size estimate
            was less than the actual result, but this is theoretically possible.
            Please send me one if you find such a matrix.

        Info [UMFPACK_FLOPS]:  the actual count of the (useful) floating-point
            operations performed.  An estimate, Info [UMFPACK_FLOPS_ESTIMATE],
            was computed by umfpack_*_*symbolic.  The estimate is guaranteed to
            be an upper bound on this flop count.  The flop count excludes
            "useless" flops on zero values, flops performed during the pivot
            search (for tentative updates and assembly of candidate columns),
            and flops performed to add frontal matrices together.

            For the real version, only (+ - * /) are counted.  For the complex
            version, the following counts are used:

                operation       flops
                c = 1/b         6
                c = a*b         6
                c -= a*b        8

        Info [UMFPACK_LNZ]: the actual nonzero entries in final factor L,
            including the diagonal.  This excludes any zero entries in L,
            although some of these are stored in the Numeric object.  The
            Info [UMFPACK_LU_ENTRIES] statistic does account for all
            explicitly stored zeros, however.  Info [UMFPACK_LNZ_ESTIMATE],
            an estimate, was computed by umfpack_*_*symbolic.  The estimate is
            guaranteed to be an upper bound on Info [UMFPACK_LNZ].

        Info [UMFPACK_UNZ]: the actual nonzero entries in final factor U,
            including the diagonal.  This excludes any zero entries in U,
            although some of these are stored in the Numeric object.  The
            Info [UMFPACK_LU_ENTRIES] statistic does account for all
            explicitly stored zeros, however.  Info [UMFPACK_UNZ_ESTIMATE],
            an estimate, was computed by umfpack_*_*symbolic.  The estimate is
            guaranteed to be an upper bound on Info [UMFPACK_UNZ].

        Info [UMFPACK_NUMERIC_DEFRAG]:  The number of garbage collections
            performed during umfpack_*_numeric, to compact the contents of the
            variable-sized workspace used by umfpack_*_numeric.  No estimate was
            computed by umfpack_*_*symbolic.  In the current version of UMFPACK,
            garbage collection is performed and then the memory is reallocated,
            so this statistic is the same as Info [UMFPACK_NUMERIC_REALLOC],
            below.  It may differ in future releases.

        Info [UMFPACK_NUMERIC_REALLOC]:  The number of times that the Numeric
            object was increased in size from its initial size.  A rough upper
            bound on the peak size of the Numeric object was computed by
            umfpack_*_*symbolic, so reallocations should be rare.  However, if
            umfpack_*_numeric is unable to allocate that much storage, it
            reduces its request until either the allocation succeeds, or until
            it gets too small to do anything with.  If the memory that it
            finally got was small, but usable, then the reallocation count
            could be high.  No estimate of this count was computed by
            umfpack_*_*symbolic.

        Info [UMFPACK_NUMERIC_COSTLY_REALLOC]:  The number of times that the
            system realloc library routine (or mxRealloc for the mexFunction)
            had to move the workspace.  Realloc can sometimes increase the size
            of a block of memory without moving it, which is much faster.  This
            statistic will always be <= Info [UMFPACK_NUMERIC_REALLOC].  If your
            memory space is fragmented, then the number of "costly" realloc's
            will be equal to Info [UMFPACK_NUMERIC_REALLOC].

        Info [UMFPACK_COMPRESSED_PATTERN]:  The number of integers used to
            represent the pattern of L and U.

        Info [UMFPACK_LU_ENTRIES]:  The total number of numerical values that
            are stored for the LU factors.  Some of the values may be explicitly
            zero in order to save space (allowing for a smaller compressed
            pattern).

        Info [UMFPACK_NUMERIC_TIME]:  The CPU time taken, in seconds.

        Info [UMFPACK_RCOND]:  A rough estimate of the condition number, equal
            to min (abs (diag (U))) / max (abs (diag (U))), or zero if the
            diagonal of U is all zero.

        Info [UMFPACK_UDIAG_NZ]:  The number of numerically nonzero values on
            the diagonal of U.

        Info [UMFPACK_UMIN]:  the smallest absolute value on the diagonal of U.

        Info [UMFPACK_UMAX]:  the smallest absolute value on the diagonal of U.

        Info [UMFPACK_MAX_FRONT_SIZE]: the size of the
            largest frontal matrix (number of entries).

        Info [UMFPACK_NUMERIC_WALLTIME]:  The wallclock time taken, in seconds.

        Info [UMFPACK_MAX_FRONT_NROWS]: the max number of
            rows in any frontal matrix.

        Info [UMFPACK_MAX_FRONT_NCOLS]: the max number of
            columns in any frontal matrix.

        Info [UMFPACK_WAS_SCALED]:  the scaling used, either UMFPACK_SCALE_NONE,
            UMFPACK_SCALE_SUM, or UMFPACK_SCALE_MAX.

        Info [UMFPACK_RSMIN]: if scaling is performed, the smallest scale factor
            for any row (either the smallest sum of absolute entries, or the
            smallest maximum of absolute entries).

        Info [UMFPACK_RSMAX]: if scaling is performed, the largest scale factor
            for any row (either the largest sum of absolute entries, or the
            largest maximum of absolute entries).

        Info [UMFPACK_ALLOC_INIT_USED]:  the initial allocation parameter used.

        Info [UMFPACK_FORCED_UPDATES]:  the number of BLAS-3 updates to the
            frontal matrices that were required because the frontal matrix
            grew larger than its current working array.

        Info [UMFPACK_NOFF_DIAG]: number of off-diagonal pivots selected, if the
            symmetric strategy is used.

        Info [UMFPACK_NZDROPPED]: the number of entries smaller in absolute
            value than Control [UMFPACK_DROPTOL] that were dropped from L and U.
            Note that entries on the diagonal of U are never dropped.

        Info [UMFPACK_ALL_LNZ]: the number of entries in L, including the
            diagonal, if no small entries are dropped.

        Info [UMFPACK_ALL_UNZ]: the number of entries in U, including the
            diagonal, if no small entries are dropped.

        Only the above listed Info [...] entries are accessed.  The remaining
        entries of Info are not accessed or modified by umfpack_*_numeric.
        Future versions might modify different parts of Info.
*/

\end{verbatim}
}

%-------------------------------------------------------------------------------
\newpage
\subsection{umfpack\_di\_solve}

% INCLUDE umfpack_di_solve
{\footnotesize
\begin{verbatim}

int umfpack_di_solve
(
    int sys,
    const int32_t Ap [ ],
    const int32_t Ai [ ],
    const double Ax [ ],
    double X [ ],
    const double B [ ],
    void *Numeric,
    const double Control [UMFPACK_CONTROL],
    double Info [UMFPACK_INFO]
) ;

int umfpack_dl_solve
(
    int sys,
    const int64_t Ap [ ],
    const int64_t Ai [ ],
    const double Ax [ ],
    double X [ ],
    const double B [ ],
    void *Numeric,
    const double Control [UMFPACK_CONTROL],
    double Info [UMFPACK_INFO]
) ;

int umfpack_zi_solve
(
    int sys,
    const int32_t Ap [ ],
    const int32_t Ai [ ],
    const double Ax [ ], const double Az [ ],
    double Xx [ ],       double Xz [ ],
    const double Bx [ ], const double Bz [ ],
    void *Numeric,
    const double Control [UMFPACK_CONTROL],
    double Info [UMFPACK_INFO]
) ;

int umfpack_zl_solve
(
    int sys,
    const int64_t Ap [ ],
    const int64_t Ai [ ],
    const double Ax [ ], const double Az [ ],
    double Xx [ ],       double Xz [ ],
    const double Bx [ ], const double Bz [ ],
    void *Numeric,
    const double Control [UMFPACK_CONTROL],
    double Info [UMFPACK_INFO]
) ;

/*
double int32_t Syntax:

    #include "umfpack.h"
    void *Numeric ;
    int32_t *Ap, *Ai ;
    int sys ;
    double *B, *X, *Ax, Info [UMFPACK_INFO], Control [UMFPACK_CONTROL] ;
    int status = umfpack_di_solve (sys, Ap, Ai, Ax, X, B, Numeric, Control,
        Info) ;

double int64_t Syntax:

    #include "umfpack.h"
    void *Numeric ;
    int64_t *Ap, *Ai ;
    int sys ;
    double *B, *X, *Ax, Info [UMFPACK_INFO], Control [UMFPACK_CONTROL] ;
    int status = umfpack_dl_solve (sys, Ap, Ai, Ax, X, B, Numeric, Control,
        Info) ;

complex int32_t Syntax:

    #include "umfpack.h"
    void *Numeric ;
    int32_t *Ap, *Ai ;
    int sys ;
    double *Bx, *Bz, *Xx, *Xz, *Ax, *Az, Info [UMFPACK_INFO],
        Control [UMFPACK_CONTROL] ;
    int status = umfpack_zi_solve (sys, Ap, Ai, Ax, Az, Xx, Xz, Bx, Bz,
        Numeric, Control, Info) ;

complex int64_t Syntax:

    #include "umfpack.h"
    void *Numeric ;
    int64_t *Ap, *Ai ;
    int sys ;
    double *Bx, *Bz, *Xx, *Xz, *Ax, *Az, Info [UMFPACK_INFO],
        Control [UMFPACK_CONTROL] ;
    int status = umfpack_zl_solve (sys, Ap, Ai, Ax, Az, Xx, Xz, Bx, Bz,
        Numeric, Control, Info) ;

packed complex Syntax:

    Same as above, Xz, Bz, and Az are NULL.

Purpose:

    Given LU factors computed by umfpack_*_numeric (PAQ=LU, PRAQ=LU, or
    P(R\A)Q=LU) and the right-hand-side, B, solve a linear system for the
    solution X.  Iterative refinement is optionally performed.  Only square
    systems are handled.  Singular matrices result in a divide-by-zero for all
    systems except those involving just the matrix L.  Iterative refinement is
    not performed for singular matrices.  In the discussion below, n is equal
    to n_row and n_col, because only square systems are handled.

Returns:

    The status code is returned.  See Info [UMFPACK_STATUS], below.

Arguments:

    int sys ;           Input argument, not modified.

        Defines which system to solve.  (') is the linear algebraic transpose
        (complex conjugate if A is complex), and (.') is the array transpose.

            sys value       system solved
            UMFPACK_A       Ax=b
            UMFPACK_At      A'x=b
            UMFPACK_Aat     A.'x=b
            UMFPACK_Pt_L    P'Lx=b
            UMFPACK_L       Lx=b
            UMFPACK_Lt_P    L'Px=b
            UMFPACK_Lat_P   L.'Px=b
            UMFPACK_Lt      L'x=b
            UMFPACK_U_Qt    UQ'x=b
            UMFPACK_U       Ux=b
            UMFPACK_Q_Ut    QU'x=b
            UMFPACK_Q_Uat   QU.'x=b
            UMFPACK_Ut      U'x=b
            UMFPACK_Uat     U.'x=b

        Iterative refinement can be optionally performed when sys is any of
        the following:

            UMFPACK_A       Ax=b
            UMFPACK_At      A'x=b
            UMFPACK_Aat     A.'x=b

        For the other values of the sys argument, iterative refinement is not
        performed (Control [UMFPACK_IRSTEP], Ap, Ai, Ax, and Az are ignored).

    Int Ap [n+1] ;      Input argument, not modified.
    Int Ai [nz] ;       Input argument, not modified.
    double Ax [nz] ;    Input argument, not modified.
                        Size 2*nz for packed complex case.
    double Az [nz] ;    Input argument, not modified, for complex versions.

        If iterative refinement is requested (Control [UMFPACK_IRSTEP] >= 1,
        Ax=b, A'x=b, or A.'x=b is being solved, and A is nonsingular), then
        these arrays must be identical to the same ones passed to
        umfpack_*_numeric.  The umfpack_*_solve routine does not check the
        contents of these arguments, so the results are undefined if Ap, Ai, Ax,
        and/or Az are modified between the calls the umfpack_*_numeric and
        umfpack_*_solve.  These three arrays do not need to be present (NULL
        pointers can be passed) if Control [UMFPACK_IRSTEP] is zero, or if a
        system other than Ax=b, A'x=b, or A.'x=b is being solved, or if A is
        singular, since in each of these cases A is not accessed.

        If Az, Xz, or Bz are NULL, then both real
        and imaginary parts are contained in Ax[0..2*nz-1], with Ax[2*k]
        and Ax[2*k+1] being the real and imaginary part of the kth entry.

    double X [n] ;      Output argument.
    or:
    double Xx [n] ;     Output argument, real part
                        Size 2*n for packed complex case.
    double Xz [n] ;     Output argument, imaginary part.

        The solution to the linear system, where n = n_row = n_col is the
        dimension of the matrices A, L, and U.

        If Az, Xz, or Bz are NULL, then both real
        and imaginary parts are returned in Xx[0..2*n-1], with Xx[2*k] and
        Xx[2*k+1] being the real and imaginary part of the kth entry.

    double B [n] ;      Input argument, not modified.
    or:
    double Bx [n] ;     Input argument, not modified, real part.
                        Size 2*n for packed complex case.
    double Bz [n] ;     Input argument, not modified, imaginary part.

        The right-hand side vector, b, stored as a conventional array of size n
        (or two arrays of size n for complex versions).  This routine does not
        solve for multiple right-hand-sides, nor does it allow b to be stored in
        a sparse-column form.

        If Az, Xz, or Bz are NULL, then both real
        and imaginary parts are contained in Bx[0..2*n-1], with Bx[2*k]
        and Bx[2*k+1] being the real and imaginary part of the kth entry.

    void *Numeric ;             Input argument, not modified.

        Numeric must point to a valid Numeric object, computed by
        umfpack_*_numeric.

    double Control [UMFPACK_CONTROL] ;  Input argument, not modified.

        If a (double *) NULL pointer is passed, then the default control
        settings are used.  Otherwise, the settings are determined from the
        Control array.  See umfpack_*_defaults on how to fill the Control
        array with the default settings.  If Control contains NaN's, the
        defaults are used.  The following Control parameters are used:

        Control [UMFPACK_IRSTEP]:  The maximum number of iterative refinement
            steps to attempt.  A value less than zero is treated as zero.  If
            less than 1, or if Ax=b, A'x=b, or A.'x=b is not being solved, or
            if A is singular, then the Ap, Ai, Ax, and Az arguments are not
            accessed.  Default: 2.

    double Info [UMFPACK_INFO] ;        Output argument.

        Contains statistics about the solution factorization.  If a
        (double *) NULL pointer is passed, then no statistics are returned in
        Info (this is not an error condition).  The following statistics are
        computed in umfpack_*_solve:

        Info [UMFPACK_STATUS]: status code.  This is also the return value,
            whether or not Info is present.

            UMFPACK_OK

                The linear system was successfully solved.

            UMFPACK_WARNING_singular_matrix

                A divide-by-zero occurred.  Your solution will contain Inf's
                and/or NaN's.  Some parts of the solution may be valid.  For
                example, solving Ax=b with

                A = [2 0]  b = [ 1 ]  returns x = [ 0.5 ]
                    [0 0]      [ 0 ]              [ Inf ]

            UMFPACK_ERROR_out_of_memory

                Insufficient memory to solve the linear system.

            UMFPACK_ERROR_argument_missing

                One or more required arguments are missing.  The B, X, (or
                Bx and Xx for the complex versions) arguments
                are always required.  Info and Control are not required.  Ap,
                Ai, Ax are required if Ax=b,
                A'x=b, A.'x=b is to be solved, the (default) iterative
                refinement is requested, and the matrix A is nonsingular.

            UMFPACK_ERROR_invalid_system

                The sys argument is not valid, or the matrix A is not square.

            UMFPACK_ERROR_invalid_Numeric_object

                The Numeric object is not valid.

        Info [UMFPACK_NROW], Info [UMFPACK_NCOL]:
                The dimensions of the matrix A (L is n_row-by-n_inner and
                U is n_inner-by-n_col, with n_inner = min(n_row,n_col)).

        Info [UMFPACK_NZ]:  the number of entries in the input matrix, Ap [n],
            if iterative refinement is requested (Ax=b, A'x=b, or A.'x=b is
            being solved, Control [UMFPACK_IRSTEP] >= 1, and A is nonsingular).

        Info [UMFPACK_IR_TAKEN]:  The number of iterative refinement steps
            effectively taken.  The number of steps attempted may be one more
            than this; the refinement algorithm backtracks if the last
            refinement step worsens the solution.

        Info [UMFPACK_IR_ATTEMPTED]:   The number of iterative refinement steps
            attempted.  The number of times a linear system was solved is one
            more than this (once for the initial Ax=b, and once for each Ay=r
            solved for each iterative refinement step attempted).

        Info [UMFPACK_OMEGA1]:  sparse backward error estimate, omega1, if
            iterative refinement was performed, or -1 if iterative refinement
            not performed.

        Info [UMFPACK_OMEGA2]:  sparse backward error estimate, omega2, if
            iterative refinement was performed, or -1 if iterative refinement
            not performed.

        Info [UMFPACK_SOLVE_FLOPS]:  the number of floating point operations
            performed to solve the linear system.  This includes the work
            taken for all iterative refinement steps, including the backtrack
            (if any).

        Info [UMFPACK_SOLVE_TIME]:  The time taken, in seconds.

        Info [UMFPACK_SOLVE_WALLTIME]:  The wallclock time taken, in seconds.

        Only the above listed Info [...] entries are accessed.  The remaining
        entries of Info are not accessed or modified by umfpack_*_solve.
        Future versions might modify different parts of Info.
*/
\end{verbatim}
}

%-------------------------------------------------------------------------------
\newpage

\subsection{umfpack\_di\_free\_symbolic}

% INCLUDE umfpack_di_free_symbolic
{\footnotesize
\begin{verbatim}

void umfpack_di_free_symbolic
(
    void **Symbolic
) ;

void umfpack_dl_free_symbolic
(
    void **Symbolic
) ;

void umfpack_zi_free_symbolic
(
    void **Symbolic
) ;

void umfpack_zl_free_symbolic
(
    void **Symbolic
) ;

/*
double int32_t Syntax:

    #include "umfpack.h"
    void *Symbolic ;
    umfpack_di_free_symbolic (&Symbolic) ;

double int64_t Syntax:

    #include "umfpack.h"
    void *Symbolic ;
    umfpack_dl_free_symbolic (&Symbolic) ;

complex int32_t Syntax:

    #include "umfpack.h"
    void *Symbolic ;
    umfpack_zi_free_symbolic (&Symbolic) ;

complex int64_t Syntax:

    #include "umfpack.h"
    void *Symbolic ;
    umfpack_zl_free_symbolic (&Symbolic) ;

Purpose:

    Deallocates the Symbolic object and sets the Symbolic handle to NULL.  This
    routine is the only valid way of destroying the Symbolic object.

Arguments:

    void **Symbolic ;       Input argument, set to (void *) NULL on output.

        Points to a valid Symbolic object computed by umfpack_*_symbolic.
        No action is taken if Symbolic is a (void *) NULL pointer.
*/

\end{verbatim}

%-------------------------------------------------------------------------------
\subsection{umfpack\_di\_free\_numeric}

% INCLUDE umfpack_di_free_numeric
{\footnotesize
\begin{verbatim}

void umfpack_di_free_numeric
(
    void **Numeric
) ;

void umfpack_dl_free_numeric
(
    void **Numeric
) ;

void umfpack_zi_free_numeric
(
    void **Numeric
) ;

void umfpack_zl_free_numeric
(
    void **Numeric
) ;

/*
double int32_t Syntax:

    #include "umfpack.h"
    void *Numeric ;
    umfpack_di_free_numeric (&Numeric) ;

double int64_t Syntax:

    #include "umfpack.h"
    void *Numeric ;
    umfpack_dl_free_numeric (&Numeric) ;

complex int32_t Syntax:

    #include "umfpack.h"
    void *Numeric ;
    umfpack_zi_free_numeric (&Numeric) ;

complex int64_t Syntax:

    #include "umfpack.h"
    void *Numeric ;
    umfpack_zl_free_numeric (&Numeric) ;

Purpose:

    Deallocates the Numeric object and sets the Numeric handle to NULL.  This
    routine is the only valid way of destroying the Numeric object.

Arguments:

    void **Numeric ;        Input argument, set to (void *) NULL on output.

        Numeric points to a valid Numeric object, computed by umfpack_*_numeric.
        No action is taken if Numeric is a (void *) NULL pointer.
*/
}

\end{verbatim}


%-------------------------------------------------------------------------------
\subsection{umfpack\_di\_defaults}

% INCLUDE umfpack_di_defaults
{\footnotesize
\begin{verbatim}

void umfpack_di_defaults
(
    double Control [UMFPACK_CONTROL]
) ;

void umfpack_dl_defaults
(
    double Control [UMFPACK_CONTROL]
) ;

void umfpack_zi_defaults
(
    double Control [UMFPACK_CONTROL]
) ;

void umfpack_zl_defaults
(
    double Control [UMFPACK_CONTROL]
) ;

/*
double int32_t Syntax:

    #include "umfpack.h"
    double Control [UMFPACK_CONTROL] ;
    umfpack_di_defaults (Control) ;

double int64_t Syntax:

    #include "umfpack.h"
    double Control [UMFPACK_CONTROL] ;
    umfpack_dl_defaults (Control) ;

complex int32_t Syntax:

    #include "umfpack.h"
    double Control [UMFPACK_CONTROL] ;
    umfpack_zi_defaults (Control) ;

complex int64_t Syntax:

    #include "umfpack.h"
    double Control [UMFPACK_CONTROL] ;
    umfpack_zl_defaults (Control) ;

Purpose:

    Sets the default control parameter settings.

Arguments:

    double Control [UMFPACK_CONTROL] ;  Output argument.

        Control is set to the default control parameter settings.  You can
        then modify individual settings by changing specific entries in the
        Control array.  If Control is a (double *) NULL pointer, then
        umfpack_*_defaults returns silently (no error is generated, since
        passing a NULL pointer for Control to any UMFPACK routine is valid).
*/
\end{verbatim}
}

\end{document}
